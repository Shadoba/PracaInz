\documentclass[polish, a4paper, 12pt, oneside]{book}
\linespread{1.3}
\usepackage[a4paper, inner=0cm, outer=0cm, top=2.7cm, right=2.4cm, bottom=2.5cm, left=2.4cm, bindingoffset=1.2cm]{geometry}
\usepackage[utf8]{inputenc}
\usepackage[T1]{fontenc}
\usepackage{lmodern}
\usepackage{microtype}
\usepackage{graphicx}
\usepackage{index}
\usepackage{babel}
\usepackage{csquotes}
\usepackage{xpatch}
\usepackage{enumitem}
\usepackage[]{impnattypo}
\usepackage{polski}
\usepackage{changepage}
\usepackage{indentfirst}
\usepackage{enumitem}
\usepackage{afterpage}
\usepackage{capt-of}
\usepackage{caption}
\usepackage[numbers]{natbib}
\usepackage{hyperref}

\bibliographystyle{unsrt}
\setlist[itemize,2]{label={$\star$}}

\usepackage{etoolbox}
\makeatletter
\patchcmd{\chapter}{\if@openright\cleardoublepage\else\clearpage\fi}{}{}{}
\makeatother

\begin{document}
	
	\begin{titlepage}
		\includegraphics[height=37.5mm]{agh_nzw_a_pl_3w_wbr_cmyk}\\
		\rule{30mm}{0pt}
		{\large\textsf{Wydział Fizyki i Informatyki Stosowanej}}\\
		\rule{\textwidth}{3pt}\\
		\rule[2ex]
		{\textwidth}{1pt}\\
		\vspace{5ex}
		\begin{center}
			{\bf\LARGE\textsf{Streszczenie pracy inżynierskiej}}\\
			\vspace{13ex}
			{\bf\Large\textsf{Bartłomiej Mucha}}\\
			\vspace{3ex}
			{\sf \small kierunek studiów:} {\bf\small\textsf{informatyka stosowana}}\\
			\vspace{7ex}
			{\bf\huge\textsf{Migracja serwisów internetowych i integracja usług w środowisku kontenerów Docker}}\\
			\vspace{14ex}
			{\sf \Large Opiekun:} {\bf\Large\textsf{dr inż. Piotr Gronek}}\\
			\vspace{22ex}
			\textsf{\bf\large\textsf{Kraków, styczeń 2020}}
		\end{center}
	\end{titlepage}
	
	\vspace*{\fill}
	\newpage
	
	\chapter{Wstęp}
	W tym rozdziale znajduje się wprowadzenie do zagadnień związanych z wirtualizacją. Znajduję się tutaj również opis celów i założeń projektu oraz co ma zostać wykonane w ramach pracy.
	
	\chapter{Podstawa Teoretyczna}
	W rozdziale opisane są zagadnienia związane z wirtualizacją, takie jak jej typy, wymagania w oprogramowania. Opisane są również występujące w ramach działania wirtualizacji i do jej działania mechanizmy.
	
	\chapter{Założenia projektowe}
	Wyjaśniony jest stan wejściowy wydziałowego systemu informatyczne oraz to jakim zmianą zostanie poddany i jaki ma być jego stan po zrealizowaniu projektu.
	
	\chapter{Opis przykładowych aplikacji}
Zostają dokładnie opisane wdrożone przykładowe serwisy, w tym ich wymagania i zasady działania oraz powiązania między sobą. Przeanalizowane są pliki budujące kontenery \textit{Docker} orz plik konfiguracyjny środowiska \textit{Docker-compose} do zarządzania kontenerami.
	 
	\chapter{Wdrożenie przykładowych oraz rzeczywistych serwisów wydziałowego systemu informatycznego}
	Zrelacjonowany zostaje etap wdrażania projektu na wydziałowy serwer oraz sprawdzenie funkcjonowania projektu na platformie serwerowej.
	 
	\chapter{Wnioski}
	W tym rozdziale znajduję się podsumowanie projektu i samego pomysłu zawartego w temacie tej pracy dyplomowej.
	\newpage
	\chapter{Kod źródłowy}
	Miejsce, w którym można odnaleźć adresy \textit{URL} do źródeł projektu i pracy inżynierskiej oraz opis zawartości dołączonej płyty cd.
	
	
\end{document}